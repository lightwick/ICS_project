\documentclass{article}

%opening
\title{Simulating Different Receivers in a \\Rayleigh Fading, MIMO Environment\\
\large Project \#2}
\author{Intelligent Communication Systems (ICS) Lab.\\노용재}
\date{Winter Intern Seminar (2023-1)}

\usepackage{kotex} % korean
\usepackage[margin=1in]{geometry} % 둘레 margin
\usepackage{matlab-prettifier}
\usepackage{amsmath}
\usepackage{graphicx} % image
\usepackage{subcaption} % subfigure
\usepackage{xcolor} % for coloring text
\usepackage{amssymb} % because, therefore symbol
\usepackage{float}

\newcommand{\bd}{\textbf} % bold
\providecommand{\abs}[1]{\lvert#1\rvert}
\graphicspath{{./img/}}
\newcommand{\sgn}{\operatorname{sgn}}
\begin{document}

\maketitle
\tableofcontents
\vspace{0.5cm}
\hrule
\vspace{0.5cm}

\section{Implementation}
\subsection{ZF}
\bd{Moore Penrose Pseudo Inverse}\\
$W_{ZF}H=I$를 만족하는 $W_{ZF}$를 찾으려고 한다. $H^H H$는 square matrix라는 사실을 이용하여 $W_{ZF}$를 구할 수 있다는 사실을 이용해 $W_{ZF}=(H^H H)^{-1}H^H$임을 알 수 있다.
\begin{gather}
	\begin{split}
		W_{ZF}H&=(H^H H)^{-1}H^H H\\
		&=(H^H H)^{-1}(H^H H) \quad (\because associative property)\\
		&=I
	\end{split}
\end{gather}
이때, $W_{ZF}$를 $H$의 \textsl{right pseudo-inverse}라고 할 수 있다.
\section{결과 및 분석}
\subsection{Simulation Result}

\section{미해결 \& 추가연구 필요 내용}
\begin{itemize}
  \item $W_{ZF}=(H^H H)^{-1}H^H$, transpose를 한다고 하더라도 동일한 결과를 가져오는 것 아닌가? 왜 zf에서 transpose가 아닌 hermitian transpose를 사용한건가?
\end{itemize}
\section[Entire Code]{Entire Code \footnote{Uploaded on https://github.com/lightwick/ICS\_project}}
\end{document}